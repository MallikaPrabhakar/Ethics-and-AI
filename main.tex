\documentclass[a4paper]{article}

%% Language and font encodings
\usepackage[english]{babel}
\usepackage[utf8x]{inputenc}
\usepackage[T1]{fontenc}

%% Sets page size and margins
\usepackage[a4paper,top=3cm,bottom=2cm,left=3cm,right=3cm,marginparwidth=2cm]{geometry}

%% Useful packages
\usepackage{amsmath}
\usepackage{graphicx}
\usepackage[colorinlistoftodos]{todonotes}
\usepackage[colorlinks=true, allcolors=blue]{hyperref}

%Document starts
\title{Ethics and Artificial Intelligence}
\author{Mallika Prabhakar \\ {2019CS50440}}
\date{October, 2020}

%Begin
\begin{document}
\maketitle


\section{Introduction}

Artificial intelligence is a growing field and has already had many applications like speech recognition software, self-driving cars and chat bots, but most people ignore or aren't simply aware of the various ethical issues it might cause. These ethical issues are urgent and we really need to pay attention to these. Some examples of ethical issues which may occur with the advent of AI in our lives are- bias and fake news, exploitation of workers, privacy, etc.

\section{Artificial Intelligence}

EU Commission Communication considers Artificial Intelligence as the machines that have the ability to make intelligent actions by taking feedback from the surroundings to accomplish a task.\cite{EPRS}

Artificial intelligence is the capability of a system that copies the working of human mind to learn from the data or examples it is fed and perform actions. Some common motives to use AI are rendering natural language and reacting to it, performing actions like driving a car, making decisions and solving problems just how a human mind would.


\section{Ethics and AI}

Ethics are the foundation of human society. They help humans to adhere to a basic set of rules or morals thereby reducing unruly or unfair behaviour which might harm a person or a community. Ethics have always been debatable. Immanuel Kant, a philosopher has said in a plain manner that one should act how he would like to be treated\cite{kant}\newline 
Artificial intelligence ethics is minimising the damage which can be caused by introducing AI in various fields like data science, medical sciences and military technologies. The serious challenges to ethics caused by AI are data privacy, bias in AI systems, replacing humans in different jobs, etc. 


\section{Ethical issues}
Here are some of the areas where ethical issues which are or will be faced by employing Artificial intelligence-
\subsection{Jobs and AI}
McKinsey's report says that a lot of people (about 800 million) will become unemployed as the robots which use Artificial intelligence will start working at their place by the year 2030\cite{McKinsey}. But if we allow the robots to do the laborious tasks, we can create new jobs which might include supervision and better roles like administrative duties. Even then, the net increase of jobs is not guaranteed and it remains an important ethical and economic issue.
\subsection{Bias in Artificial Intelligence}
Artificial Intelligence can not be trusted to be neutral. It is so because of the data which is mainly available for it to process. One such example in computer vision is of photos service provided by google which uses Artificial Intelligence to perform various types of identifications in images. The major ethical issue posed here is the wrong predictions on different races. The bias in data also makes AI biased like it would predict black people to be  more often in a future criminal predictor.
\subsection{Privacy and Security}
AI systems can be used to cause damage just like other resources if used maliciously. Cyber-security will become an even more important issue as people would try to extract more and more private data . For example, Intelligent personal assistants of corporate firms might leak out data hence regulatory laws must be applied.
\subsection{Democracy}
AI is a dangerous tool and if it is used by a small amount of economically and politically powerful people, it will be easier to manipulate people and influence the government. Miscommunication of data and spreading fake news is also possible which will heavily affect the decision making of people.
\subsection{Human psychology}
One of the main thing which makes humans humane is how they interact with the people around them. Artificial intelligence has been continuously improving in this field. As time progresses, they will be used in different ways like as baby sitters, nurses, teachers, etc. but this will highly impact the human-human interaction. Moreover, the bots can be used for manipulation and threats to humans.
\subsection{Environment}
 AI will likely increase the demand of resources. The increase in demand will cause a lot of excavation at digging sites which might turn hazardous and depletion of resources. The amount of electronic waste will also increase and so will the fuel requirement which will further impact the planet. Alternatively, AI can help us by managing wastes and creating conservation strategies.
\section{Summary}
Artificial Intelligence will play a big role in modelling our future and it comes with a lot of risks and exploitable downsides. We need to use it in the correct manner by minimising any risk of it causing harm to the society. Developing AI while keeping ethics into account is the need of the hour. There should be proper rules laid out so that the corporate or anyone applying Artificial Intelligence can not misuse it to endanger the rights of people.

\bibliographystyle{alpha}
\bibliography{sample}
\url{https://www.ibm.com/cloud/learn/what-is-artificial-intelligence}
\newline
\url{https://www.weforum.org/agenda/2016/10/top-10-ethical-issues-in-artificial-intelligence/}
\end{document}